\section{}
\label{sec:p1}
%%%%%%%%%%%%%%%%%%%%%%%%%%%%%%%%%%%%%%%%%%%%%%%%%%%%%%%%%%%%%%%%%%%%%%
\paragraph{a)}
The '\textit{seismic.dat}' data contains seismic data on a regular grid $(75\times75)$, $\matr{L}_D$, denoted as $\{d(\vect{x});\vect{x}\in \matr{L}_D\};d(\vect{x})\in \R$.
The seismic data has the likelihood model given by the equation
\begin{equation}
    [d_i\given\vect{l}] = \begin{cases}
                                0.02 + U_i, & l_i = 0 \Rightarrow \mathrm{sand} \\
                                0.08 + U_i, & l_i = 1 \Rightarrow \mathrm{shale}
                             \end{cases}
                            , i = 1, 2, ..., n,
    \label{eq:likelihood}
\end{equation}

where $U_i \overset{\mathrm{iid}}{\sim}\N\{0,0.06^2\}$. This means that we can rewrite Equation \ref{eq:likelihood} to 
\begin{equation}
    [d_i\given\vect{l}] = \begin{cases}
                                A_i\overset{\mathrm{iid}}{\sim}\N\{0.02,0.06^2\}, & l_i = 0 \Rightarrow \mathrm{sand} \\
                                B_i\overset{\mathrm{iid}}{\sim}\N\{0.08,0.06^2\}, & l_i = 1 \Rightarrow \mathrm{shale}
                             \end{cases}
                            , i = 1, 2, ..., n.
    \label{eq:likelihood2}
\end{equation}
The likelihood model $p(\vect{d}\given \vect{l})$ is then given by 
\begin{equation}
    \begin{array}{rcl}
        [\vect{d}\given\vect{l}] \sim p(\vect{d}\given\vect{l}) & = & \prod\limits_{i=1}^n p(d_i\given\vect{l}) \\
         & = & \prod\limits_{i=1}^n \begin{cases}
                                        \N\{0.02,0.06^2\}, & l_i = 0 \\
                                        \N\{0.08,0.06^2\}, & l_i = 1
                                    \end{cases}\\
         & \propto & \exp\left\{\sum\limits_{l_i = 0} \frac{(d_i - 0.2)^2}{2\cdot 0.06^2} + \sum\limits_{l_i = 1} \frac{(d_i - 0.08)^2}{2\cdot 0.06^2}\right\}
    \end{array}.
    \label{eq:likelihood3}
\end{equation}

\begin{figure}
    \centering
    \includegraphics[width=0.7\textwidth]{figures/boxplot_bili.pdf}
    \caption{A boxplot of the bilirubin data. On the y-axis the logarithm of the concentration of bilirubin in the blood of three different persons that are grouped on the x-axis.}
    \label{fig:boxplot_bili}
\end{figure}

%%%%%%%%%%%%%%%%%%%%%%%%%%%%%%%%%%%%%%%%%%%%%%%%%%%%%%%%%%%%%%%%%%%%%%
\paragraph{b)}


%%%%%%%%%%%%%%%%%%%%%%%%%%%%%%%%%%%%%%%%%%%%%%%%%%%%%%%%%%%%%%%%%%%%%%
\paragraph{c)}


%%%%%%%%%%%%%%%%%%%%%%%%%%%%%%%%%%%%%%%%%%%%%%%%%%%%%%%%%%%%%%%%%%%%%%
\paragraph{d)}

